\fancyfoot{ }
\begin{abstract}

针对小样本问题,首先构建深层残差网络来实现Relation Network模型,并在miniImageNet数据集中实现小样本的分类,
记录分类的准确率。以此为目标,进行并行加速实验的设计与实现。

首先给出实验环境,包括:1.CPU、GPU和内存等硬件设备信息;2.操作系统和开发工具等软件及版本,并简述了各个软件负责的功能;
3.实验所需的实验参数及取值。而后分别从数据并行的方法、并行加载数据的角度出发
,阐述了本次实验中并行算法的设计思想,并对并行实现的原理进行了解析,包括多进程的工作流程和通信方法等。

而后针对并行加载数据进行对比实验,列出了在不同数量进程下的加速比与并行效率,对最终结果进行分析并得到了在进程数
和处理器核心数近似相等的情况下,加速比效果较为良好的结论;
针对DataParallel和DistributedDataParallel两种不同的并行方式,进行了对比实验后取得加速比,
并结合对并行原理的解析,得到了DistributedDataParallel优于DataParallel的结论。
之后更改实验参数,分析实验在不同参数下取得的结果并得到:
并行方法的适用场合和实验参数如何取值等相关结论。

最后,在附录中列出了实验过程中遇到的相关问题和对应的解决方案,并给出实验所用的源程序。

\end{abstract}

\textbf{关键词:} 深度学习,单机多卡,数据并行
\newpage
\fancyfoot[C]{\bfseries\thepage}