\section{实验环境}

实验的成功进行离不开软件件配置,本章分别介绍实验所依赖的的硬件设备和开发环境及版本,
之后介绍本次实验所选用的参数。

\subsection{硬件设备}

\begin{enumerate}[itemsep=-1ex,partopsep=1ex]
    \item CPU:2个Intel(R) Xeon(R) Gold 5115 CPU \@ 2.40GHz,10核心20线程。
    \item GPU:4路Tesla P40,每路显存容量22GB,共88GB。
    \item 内存:128GB。
    \item 外存:520TB可用,已用15TB。
\end{enumerate}

\subsection{软件环境与版本}

\begin{enumerate}[itemsep=-1ex,partopsep=1ex]
    \item 运行程序的系统为 CentOS Linux release 7.3.1611。
    \item 开发程序的系统为 Arch 5.9.6。
    \item python:3.8.2:作为开发语言。
    \item pytorch:1.6.0:作为模型实现的工具,借助其提供的API实现并行。
    \item ssh:OpenSSH\_8.3p1, OpenSSL 1.1.1h,实现远程登录功能,即本地登录到服务器。
    \item scp:实现文件传输功能,将本地文件传输至服务器,或将服务器文件传输至本地。
\end{enumerate}

在进入服务器后,不能在根目录下直接配置实验环境,否则第三方库之间的版本依赖、冲突等问题会对他人的程序造成一定影响。
因此,首先使用conda创建虚拟环境,在conda activate进入虚拟环境后配置本次实验所需的依赖库。对于本实验,
进入虚拟环境后pip install torch torchvision即可。

\subsection{参数设置}

\begin{enumerate}[itemsep=-1ex,partopsep=1ex]
    \item n\_way = 5, 5个类为一组支持集。
    \item k\_query = 1,查询集只含有1个类。
    \item k\_shot = 1,支持集和查询集中每个类中只有一个样本。
    \item batchsz = 30,每一组训练的数据中由30组支持集和查询集组成。
\end{enumerate}

在不同的对比实验中,除需对比的参数外,其余实验参数需要保持一致,否则程序的执行时间会相差很远,无法得到准确的加速比。