\section*{附录A:遇到问题及解决方案}
\addcontentsline{toc}{section}{附录A:遇到问题及解决方案}

\subsection*{数据无法序列化}
\addcontentsline{toc}{section}{A.1:数据无法序列化}

在深度学习中,常常需要把训练好的模型或使用数据对象存储起来,这样在训练或预测时直接将模型读出,
而不需要重新读取数据,这样就大大节约了时间。Python提供的pickle库就很好地解决了这个问题,
它可以序列化对象并保存到磁盘中,并在需要的时候读取出来。

在加载数据时,使用多进程加载数据,并设置num\_workers来指定进程的数量。
在这样形式下获取的数据可以在DataParallel方法中训练,
并没有错误。但在DistributedDataParallel方法训练时,提示有数据无法序列化的原因。

在仔细阅读DistributedDataParallel和DataLoader的官方文档后发现了原因,
因为num\_workers会开启多进程读取数据,且序列化数据。而DistributedDataParallel
训练数据时,也会序列化数据。导致在DistributedDataParallel训练时会有两次序列化数据,
因此产生错误。所以,在DistributedDataParallel训练方式中,只需要
将num\_workers设定为0即可。

\subsection*{显存溢出错误}
\addcontentsline{toc}{section}{A.2:显存溢出错误}

深度学习的网络参数和数据量往往较大,因此常常发生显存溢出错误,且因为DistributedDataParallel独特的并行机制,
会需要更大的显存空间,所以需要一些设计技巧来规避显存溢出问题。

\begin{enumerate}[leftmargin=0pt,itemindent=3.5\ccwd]
    \item 为避免浪费服务器的内存,可以把模型和数据通过.cuda()方法放到CUDA中在使用。
    \item 通常在训练结束后想要绘制loss的图像,都是将loss添加到一个列表中保存。
    但在保存前可以使用detach方法将loss的数值与计算图解锁,然后仅保存其数值。
    否则会将计算图添加到列表中,造成内存的浪费。
    \item 使用numpy操作数据,而不是列表。python的列表存储了对象引用地址和对象本身,因此对象在内存中并不连续,
    通过引用地址来访问对象。numpy在会开辟连续的内存空间来存储对象,因此只需要存储一个引用。相比之下,在内存空间、
    访问效率等两个角度而言,numpy比列表更具优势。
    \item DistributedDataParallel需要更大的显存空间去存储参数状态、自动求导的钩子、用于通信的参数桶
    和Reducer。所以在每一次训练过程中,不能加载过大的的训练数据。
\end{enumerate}

\subsection*{数据泄漏}
\addcontentsline{toc}{section}{A.3:数据无法共享}

在DistributedDataParallel并行方式中,显示rank1无法获取rank0数据的错误,可以分析得到出错的原因是
DistributedDataParallel的机制限制了多进程之间不能共享同一张卡。

在发现问题出错的原因后,定位到代码前后查找错误来源。在训练过后的验证阶段,为计算网络的准确率,
额外声明了一个计算图外的变量,所以这个变量并没有准确的归属,可能在0号显卡,也可能在1号显卡。
在AllReduce通信时,发现一张显卡有这个变量,而其余的显卡没有这个变量,所以导致此类错误。
因此需要将计算图外的变量放到每一张显卡中来避免此类错误,而显卡使用进程编号rank标记,
所以只需要对数据使用.to(rank)方法即可。在采用上述方法后,错误消失。

\subsection*{无法迭代0维张量}
\addcontentsline{toc}{section}{A.4:无法迭代0维张量}

在训练的开始时,报错信息为:无法迭代0维度的数据,定位到出错代码位置,打印相关张量后发现数据维度并不是0。
但解释器仍然提示无法迭代0维向量。因此扩大debug范围,观察出错代码的函数,发现函数需要传入5个参数,而在调用
时少传入一个参数只传入了4个参数,所以产生无法迭代0维张量的错误。在补全参数后,此错误消失,网络可以被正确训练。
因此解释器的报错提示并不一定准确,还需要对出错代码进行实际查验。

\section*{附录B:实验程序相关文件}
\addcontentsline{toc}{section}{附录B:实验程序相关文件}

使用ssh连接到服务器后并进入虚拟环境后,按照前文所述进行实验。
因代码量较大,已达千行左右,可在Github
\footnote{\scriptsize\ttfamily{https://github.com/muyuuuu/Algorithm/tree/master/meta-learning/Metric-based/Relation-Netowrk}}获取完整程序。
附录只对各个模块的核心程序进行展示。

并行加载数据方法的核心程序:
\begin{code}
    \lstinputlisting[language=Python]{code/data.py}
\end{code}

DistributedDataParallel并行方法核心程序:
\begin{code}
    \lstinputlisting[language=Python]{code/ddp.py}
\end{code}

DataParallel并行方法核心程序:
\begin{code}
    \lstinputlisting[language=Python]{code/dp.py}
\end{code}

Relation Network核心程序:
\begin{code}
    \lstinputlisting[language=Python]{code/network.py}
\end{code}

特征提取层中的一组模块:
\begin{code}
    \lstinputlisting[language=Python]{code/feature.py}
\end{code}

Relation Network最后用于计算Relatoin Score的全连接层:
\begin{code}
    \lstinputlisting[language=Python]{code/fn.py}
\end{code}